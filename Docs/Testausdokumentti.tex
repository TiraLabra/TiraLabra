%%%%%%%%%%%%%%%%%%%%%%%%%%%%%%%%%%%%%%%%%
% University/School Laboratory Report
% LaTeX Template
% Version 3.1 (25/3/14)
%
% This template has been downloaded from:
% http://www.LaTeXTemplates.com
%
% Original author:
% Linux and Unix Users Group at Virginia Tech Wiki 
% (https://vtluug.org/wiki/Example_LaTeX_chem_lab_report)
%
% License:
% CC BY-NC-SA 3.0 (http://creativecommons.org/licenses/by-nc-sa/3.0/)
%
%%%%%%%%%%%%%%%%%%%%%%%%%%%%%%%%%%%%%%%%%

%----------------------------------------------------------------------------------------
%	PACKAGES AND DOCUMENT CONFIGURATIONS
%----------------------------------------------------------------------------------------

\documentclass{article}
\usepackage[finnish]{babel} %ääkköset
\usepackage[utf8]{inputenc}
\usepackage[T1]{fontenc}

\usepackage[version=3]{mhchem} % Package for chemical equation typesetting
\usepackage{siunitx} % Provides the \SI{}{} and \si{} command for typesetting SI units
\usepackage{graphicx} % Required for the inclusion of images
\usepackage{natbib} % Required to change bibliography style to APA
\usepackage{amsmath} % Required for some math elements 

\setlength\parindent{0pt} % Removes all indentation from paragraphs

\renewcommand{\labelenumi}{\alph{enumi}.} % Make numbering in the enumerate environment by letter rather than number (e.g. section 6)

%\usepackage{times} % Uncomment to use the Times New Roman font

%----------------------------------------------------------------------------------------
%	DOCUMENT INFORMATION
%----------------------------------------------------------------------------------------

\title{Testausdokumentti: elektronirakenteen optimointi atomiorbitaalittomaan tiheysfunktionaaliteorian avulla} % Title

\author{Markus \textsc{Kaukonen} % Author name
}

\date{\today} % Date for the report


\begin{document}

\maketitle % Insert the title, author and date
\hspace{1cm} \texttt{email: markus.kaukonen@iki.fi, opiskelijanumero: 010974524}

\newpage
% If you wish to include an abstract, uncomment the lines below
% \begin{abstract}
% Abstract text
% \end{abstract}

%----------------------------------------------------------------------------------------


\section{Mitä on testattu, miten tämä tehtiin}
Tehtiin unit testit kaikille tärkeimmille funkioille jotka ovat
paketeissa energia.py ja gridi.py (muissa paketeissa on
minimointirutiinija jotka ovat opettavaisia tässä työssä mutta jotka
kannattaa korvata scipy:n minimointialgoritmeilla.  Testaus
toteutettiin Pythonin unittest moduulilla. Sen dokumentaatio on
osoitteessa:
https://docs.python.org/2/library/unittest.html\#type-specific-methods.

Yksikkötestit ajetaan suorittamalla ohjelmat './testaa\_energiat.py' ja
'testaa\_gridi.py'
hakemistossa 'TiraLabra/Tiralabra/python\_unit\_tests'.


\section{Minkälaisilla syötteillä testaus tehtiin (vertailupainotteisissa töissä tärkätä)}
Testeissä käytettiin 2-dimensionaalista dataa joka luettiin tiedosta
'alkuarvot.txt\_5x5x0'. Siinä on mahdollisimman pieni 2d-laskentagridi
(3x3 sisäpistettä) ja yksi +1|e| varauksinen ydin gridin kohdassa (2,2).

\section{Miten testit voidaan toistaa}
Testit voidaan toistaa ajamalla komento './testaa\_energiat.py'
ja 'testaa\_gridi.py',hakemistossa 'TiraLabra/Tiralabra/python\_unit\_tests'.

\section{Ohjelman toiminnan empiirisen testauksen tulosten esittäminen graafisessa muodossa.}
Ohjelman tuottamat elektronitiheydet ovat kvalitatiivisesti oikeita,
koska elektronitiheys keskittyy positiivisten ytimien
ympärille. Tarkempaa vertailua olisi voinut tehdä laskemalla
systeemien välisiä energiaeroja (esimerkiksi paljonko voitetaan
energiassa kun tuodaan +1|e| ydin systeemiin. Tätä energiaeroa olisi
voinut verrata muiden kvanttikemiallisten ohjelmien
tuloksiin. Esimerkiksi hyvä vertailu ohjelma olisi GPAW
(https://wiki.fysik.dtu.dk/gpaw), jossa voidaan tehdä pistemäisiä
pseudopotentiaaleja, jotka ovat siis samanlaisia kuin tässä työssä
käytetyt pistevaraukset.


\end{document}
