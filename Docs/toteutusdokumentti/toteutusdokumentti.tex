\documentclass[a4paper,12pt, titlepage]{article}
\usepackage{amssymb,amsthm,amsmath} %ams
\usepackage[finnish]{babel} %suomenkielinen tavutus
\usepackage[T1]{fontenc} %skanditavutus
\usepackage[utf8]{inputenc} % skandit utf-8 koodauksella

\usepackage{graphicx} %dokumentti sisältää eps-muotoisia kuvia

\usepackage{changepage}
\usepackage{array}
\usepackage{tabularx}

\usepackage{hyperref}
\hypersetup{
    colorlinks=false,
    pdfborder={0 0 0},
}

\usepackage{float}
\restylefloat{table}


\linespread{1.24} %riviväli 1.5
\sloppy % Vähentää tavutuksen tarvetta, "leventämällä" rivin keskellä olevia välilyöntejä.
\begin{document}

\begin{titlepage}
    \begin{center}
        \vspace*{1cm}
        
        \LARGE
        \textbf{Toteutusdokumentti}
        
        \vspace{0.5cm}
        \Large
        Aineopintojen harjoitustyö: Tietorakenteet ja algoritmit (alkukesä)
        
        \vspace{1.5cm}
        
        \large
        \textbf{Sami Korhonen} \\
        \text{014021868} \\
        \text{sami.korhonen@helsinki.fi}
        
		\vfill        
        \normalsize
        Tietojenkäsittelytieteen laitos\\
        Helsingin yliopisto\\
		\large        
        \today
        
    \end{center}
\end{titlepage}


%\tableofcontents
%\newpage

\section*{Ohjelman yleisrakenne}
\subsection*{Luokka: App}
\subsubsection*{Metodit}
\subsection*{Luokka: Tila}
\subsubsection*{Metodit}
\subsection*{Muut luokat}
\subsubsection*{Metodit}
\subsection*{Tietorakenteet}

\section*{Saavutetut aika- ja tilavaativuudet}

\section*{Suorituskyky- ja O-analyysivertailu}

\section*{Puutteet ja parannusehdotukset}

\begin{thebibliography}{9}

\bibitem{fanslau-bortfeldt}
  Tobias Fanslau, Andreas Bortfeldt, 2008 \newline
  \emph{A Tree Search Algorithm for Solving the Container Loading Problem }\newline
  http://www.fernuni-hagen.de/wirtschaftswissenschaft/download/beitraege/db426.pdf\newline
  viitattu 18. toukokuuta 2014

\end{thebibliography}

\end{document}