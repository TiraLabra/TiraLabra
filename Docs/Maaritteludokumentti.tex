%%%%%%%%%%%%%%%%%%%%%%%%%%%%%%%%%%%%%%%%%
% University/School Laboratory Report
% LaTeX Template
% Version 3.1 (25/3/14)
%
% This template has been downloaded from:
% http://www.LaTeXTemplates.com
%
% Original author:
% Linux and Unix Users Group at Virginia Tech Wiki 
% (https://vtluug.org/wiki/Example_LaTeX_chem_lab_report)
%
% License:
% CC BY-NC-SA 3.0 (http://creativecommons.org/licenses/by-nc-sa/3.0/)
%
%%%%%%%%%%%%%%%%%%%%%%%%%%%%%%%%%%%%%%%%%

%----------------------------------------------------------------------------------------
%	PACKAGES AND DOCUMENT CONFIGURATIONS
%----------------------------------------------------------------------------------------

\documentclass{article}
\usepackage[finnish]{babel} %ääkköset
\usepackage[utf8]{inputenc}
\usepackage[T1]{fontenc}
\usepackage{listings}

\usepackage[version=3]{mhchem} % Package for chemical equation typesetting
\usepackage{siunitx} % Provides the \SI{}{} and \si{} command for typesetting SI units
\usepackage{graphicx} % Required for the inclusion of images
\usepackage{natbib} % Required to change bibliography style to APA
\usepackage{amsmath} % Required for some math elements 

\setlength\parindent{0pt} % Removes all indentation from paragraphs

\renewcommand{\labelenumi}{\alph{enumi}.} % Make numbering in the enumerate environment by letter rather than number (e.g. section 6)

%\usepackage{times} % Uncomment to use the Times New Roman font

%----------------------------------------------------------------------------------------
%	DOCUMENT INFORMATION
%----------------------------------------------------------------------------------------

\title{Määrittelydokumentti: elektronirakenteen optimointi atomiorbitaalittomaan tiheysfunktionaaliteorian avulla} % Title

\author{Markus \textsc{Kaukonen} % Author name
}

\date{\today} % Date for the report


\begin{document}

\maketitle % Insert the title, author and date
\hspace{1cm} \texttt{email: markus.kaukonen@iki.fi, opiskelijanumero: 010974524}

\newpage
% If you wish to include an abstract, uncomment the lines below
% \begin{abstract}
% Abstract text
% \end{abstract}

%----------------------------------------------------------------------------------------

\section{Johdanto}
Tarkoituksena on ratkaista pienen 2-dimensionaalisen alueen
elektronien Schrödingerin yhtälö. Ratkaisussa käytetään
tiheysfunktionaaliteoriaa, jossa kokonaisenergia riippuu vain
elektronitiheyksistä. Tässä työssä käytetään sellaista
tiheysfunktionaaliterorian versiota jossa aaltofunktioita ei käytetä
lainkaan vaan pitäydytään ainoastaan elektronitiheydessä
[\cite{ke2013angular}]. Menetelmän tarkkuutta on saatu dramaattisesti
parannetrua aivan viime aikoina jakamalla eletronitiheys elektronin
kulmaliikemäärän maukaisiin komponentteihin (eli s, p ja d elektronit
erikseen).

\section{Työssä käytettäviä algoritmeja ja tietorakenteita}
Energian minimointiin ja elektronitiheyden muuttamiseen käytetään sekä
steepest descent että Monte Carlo algoritmeja ja niiden tehokkuutta
vertaillaan. Lisäksi Poissonin yhtälö (yhtälö \ref{poisson}) ratkaistaan
erikseen muuttamalla yhtälö diskreetiksi ongelmaksi gridissä, eli
toiselle derivaatan arvot lasketaan gridin naapuripisteiden avulla.

Keskeinen tietorakenne tässä työssä on gridi objekti. Siihen kuuluu
laskenta gridin määrittely, laskentagridin reunaehdot ja erilaiset
laskutoimitukset laskentagridissä. Käytännössä gridi esittää pientä
avaruudellista tilavuutta joka on jaettu pieniin osatilavuuksiin ja
halututtujen muuttujien arvot lasketaan gridin joka
pisteessä. Tällaisia gridi-objekteja ovat esim. elektronitiheys ja
Hartree potentiaali(yhtälö \ref{hartree}).

\subsection{Monte Carlo algoritmi energian minimoiseksi}
Tässä työssä Monte Carlo algoritmi on tehty seuraavalla tavalla.
\begin{enumerate}
\item Lasketaan systeemin kokonaisenergia.
\item Arvotaan pieni tiheyden muutos jokaiselle gridin
  sisäpisteelle. Arvonta suoritetaan siten että elektronien
  kokonaismäärä säilyy.
\item Lasketaan systeemin kokonaisenergia uudella elektronitiheydellä.
\item Jos energia on alentunut hyväksytään uusi elektronitiheys ja
  siirrytään arpomaan uusi elektronitiheys (kohta b).
\item Jos energia on noussut uusi elektronitiheys hyväksytään
  todennäköisyydellä joka on otettu Bolztmannin jakaumasta
  (hyväksymisprofiilia voi muuttaa muuttamalla systeemin lämpötilaa).
\item Lopetetaan jos konvergoinut, muuten palataan kohtaan b ja
  arvotaan uusi elektronitiheys.
\end{enumerate}

\subsection{Steepes Descent algoritmi energian minimoiseksi}
Tässä työssä Steepest Descent algoritmi on tehty seuraavalla tavalla.
\begin{enumerate}
\item Lasketaan systeemin kokonaisenergia.
\item Lasketaan tiheyden gradientti jokaisessa gridin sisäpisteessä.
\item Muutetaan tiheyttä gradientin suunnassa kunnes löydetään minimi
  tässä suunnassa. Minimointi gradientin suunnassa tehdään siten että
  ensin etsitään minim karkeasti siirtymällä kiinteitä askeleita
  gradientin suuntaan. Tämän jälkeen minimi haarukoidaan tarkasti
  (Golden section telescoping [\cite{kiusalaas2013numerical}]
\end{enumerate}


\section{Ratkaistava ongelma. Valittujen algoritmien perustelu.}
Minimoidaan systeemin kokonaisenergiaa joka riippuu elektronitiheydestä $\rho$ seuraavasti:
\begin{equation}
E[\rho] = 
\int \mathrm{d}^3\vec{r} T_s[\rho(\vec{r})] + 
\int \mathrm{d}^3\vec{r} E_{xc}[\rho(\vec{r})] + 
\frac{1}{2} \iint \mathrm{d}^3\vec{r}\mathrm{d}^3\vec{r'}
\frac{\rho(\vec{r}) \rho(\vec{r'})}{|\vec{r}-\vec{r'}|} +
\sum^i \int \mathrm{d}^3\vec{r} 
\frac{\rho(\vec{r}) Z_i }{|\vec{r}-\vec{R_i}|} 
\end{equation}
Yhtälön kaksi ensimmäistä termiä ovat elektronien kineettinen energia
ja vaihtokorrelaatio energia. Ne ovat edelleen tiedeyhteisön
aktiivisen kehityksen kohteena. Tässä työssä käytetään mahdollisimman
yksinkertaisia versioita. Otetaan ns. Thomas-Fermi approksimaation
kineettiselle energialle eli
\begin{equation}
T_s[\rho(\vec{r})] = C \int \mathrm{d}^3\vec{r} \rho(\vec{r})^{5/3})
\end{equation}
jossa vakio $C=\frac{3}{10}(3\pi^2)^{2/3} = 2.871$
[\cite{march1957thomas}].  Samoin käytetään mahdollisimman
yksinkertaista vaihto-korrelaation energia funktionaalia:
\begin{equation}
E_{xc}[\rho(\vec{r})] = \frac{-3}{4}(\frac{3}{4})^{1/3}\int
\mathrm{d}^3\vec{r} \rho(\vec{r})^{4/3}
\end{equation}.
Yhtälöt ovat siinä muodossa että on käytetty atomiyksiköitä (atomic
units).

Yksinkertaisuuden vuoksi tässä työssä ei käytetä periodisia
reunaehtoja, vaan avaruus loppuu gridin reunaan. Reunaehdoksi on
valittu se että elektronitiheys menee nollaan gridin
reunalla. Integraalit lasketaan summaamalla gridi-pisteiden
ylitse. Mahdollisesti osa elektronitiheyden laskemisesta tehdään
ratkaisemalla Poissonin yhtälö:
\begin{equation}
\label{poisson}
\nabla^2 V_h = -4 \pi \rho(\vec{r}),
\end{equation}
missä
\begin{equation}
\label{hartree}
V_h (\rho, \vec{r}) = \int \mathrm{d}^3\vec{r'} 
\frac{\rho(\vec{r'}) }{|\vec{r}-\vec{r'}|}.
\end{equation}.

Jotta elektronien lukumäärä pysyy vakiona, energiaan $E$ täytyy
modifioida lisätermillä:
\begin{equation}
E=E-\mu (\int \mathrm{d}^3\vec{r}\rho(\vec{r})-N_e),
\end{equation}
jossa $\mu$ on systeemin kemiallinen potentiaali ja 
$N_e$ elektronien lukumäärä.


\section{Ohjelman saamat syötteet ja miten niitä käytetään}
Ohjelmalle annetaan syötteenä käytetyn gridin gridipisteiden määrä x-,
y- ja z- suunnissa. Lisäksi annetaan gridin pituudet atomiyksiköissä
x-, y- ja z- suunnissa. Lopulta annetaan atomiytimien paikat ja ytimen
protonien määrä (eli mistä alkuaineesta on kysymys). Esimerkki
syötetiedosto on näytetty alla:
\begin{small}
\begin{lstlisting}[frame=single]  % Start your code-block

16         # laskenta gridipisteiden lkm x suunnassa 
16         # laskenta gridipisteiden lkm y suunnassa 
0          # laskenta gridipisteiden lkm z suunnassa (0==2d)
2.0        # elektronien kokonaismaara (> 0)
2          # atomiydinten lukumaara
3 3 0 1.0  # ytimen 1 paikka gridissa (x,y,z), varaus (> 0)
8 2 0 1.0  # ytimen 2 paikka gridissa (x,y,z), varaus (> 0)
\end{lstlisting}
\end{small}


\section{Ohjelman tavoitteena olevat aika- ja tilavaativuudet}
Ohjelman testisysteemistä olisi tarkoitus rakentaa sellainen että
raskaimmatkin laskut kestäisivät luokkaa 1 CPUh. Tarkoitus on tehdä
laskenta 2d-systeemissä ja vain testata että ohjelma toimii myös
kolmessa ulottuvuudessa.

%----------------------------------------------------------------------------------------
%	BIBLIOGRAPHY
%----------------------------------------------------------------------------------------

\bibliographystyle{apalike}

\bibliography{Maarittelydokumentti}

%----------------------------------------------------------------------------------------


\end{document}
