\documentclass[12pt]{article}

\newcommand{\iii}{\indent \indent \indent}	

\title{Tietorakenteet harjoitusty\"{o}: Hakupuut}
\author{Eero S\"{a}\"{a}ksvuori}
\date{Elokuu 28, 2013}

\begin{document}

\maketitle

\noindent


\section{Ohjelman yleisrakenne}

Ohjelma on toteutettu k\"{a}ytt\"{a}en rajapintaa, jossa on jokaiselle
puulle ominaiset operaatiot: lis\"{a}\"{a}, poista, hae. Bin\"{a}\"{a}rihakupuu
on toteutettu pohjaluokkana, josta muut luokat periv\"{a}t tarvittavia
metodeja sek\"{a} laajentavat mielens\"{a} mukaan. Puut koostuvat solmuista,
jossa m\"{a}\"{a}ritell\"{a}\"{a}n sen arvo, vanhempi sek\"{a} oikea ett\"{a} vasen. 

Punamustapuun toteutus vaati nil-solmun, jonka takia bin\"{a}\"{a}rihakupuun metodit eiv\"{a}t toimineet. T\"{a}m\"{a}n takia punamustapuu ei peri bin\"{a}\"{a}rihakupuuta, vaan on
oma toteutuksensa.

Suorituskyky luokka suorittaa CLIn jossa voi asettaa erilaisia suorituskyky\"{a}
mittaavia testej\"{a}.


\section{Saavutetut aika- ja tilavaativuudet}
\section{Suorituskyky- ja O-analyysivertailu}
\section{Ty\"{o}n puutteet ja parannusehdotukset}
\section{L\"{a}hteet}

\end{document}
