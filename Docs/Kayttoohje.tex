%%%%%%%%%%%%%%%%%%%%%%%%%%%%%%%%%%%%%%%%%
% University/School Laboratory Report
% LaTeX Template
% Version 3.1 (25/3/14)
%
% This template has been downloaded from:
% http://www.LaTeXTemplates.com
%
% Original author:
% Linux and Unix Users Group at Virginia Tech Wiki 
% (https://vtluug.org/wiki/Example_LaTeX_chem_lab_report)
%
% License:
% CC BY-NC-SA 3.0 (http://creativecommons.org/licenses/by-nc-sa/3.0/)
%
%%%%%%%%%%%%%%%%%%%%%%%%%%%%%%%%%%%%%%%%%

%----------------------------------------------------------------------------------------
%	PACKAGES AND DOCUMENT CONFIGURATIONS
%----------------------------------------------------------------------------------------

\documentclass{article}
\usepackage[finnish]{babel} %ääkköset
\usepackage[utf8]{inputenc}
\usepackage[T1]{fontenc}
\usepackage{listings}

\usepackage[version=3]{mhchem} % Package for chemical equation typesetting
\usepackage{siunitx} % Provides the \SI{}{} and \si{} command for typesetting SI units
\usepackage{graphicx} % Required for the inclusion of images
\usepackage{natbib} % Required to change bibliography style to APA
\usepackage{amsmath} % Required for some math elements 

\setlength\parindent{0pt} % Removes all indentation from paragraphs

\renewcommand{\labelenumi}{\alph{enumi}.} % Make numbering in the enumerate environment by letter rather than number (e.g. section 6)

%\usepackage{times} % Uncomment to use the Times New Roman font

%----------------------------------------------------------------------------------------
%	DOCUMENT INFORMATION
%----------------------------------------------------------------------------------------

\title{Käyttöohje: elektronirakenteen optimointi atomiorbitaalittomaan tiheysfunktionaaliteorian avulla} % Title

\author{Markus \textsc{Kaukonen} % Author name
}

\date{\today} % Date for the report


\begin{document}

\maketitle % Insert the title, author and date
\hspace{1cm} \texttt{email: markus.kaukonen@iki.fi, opiskelijanumero: 010974524}

\newpage
% If you wish to include an abstract, uncomment the lines below
% \begin{abstract}
% Abstract text
% \end{abstract}

%----------------------------------------------------------------------------------------

\section{Käyttöohje}
Ohjelmaa vaatii toimiakseen pythonin, uudehkon scipy: ja numpy:n sekä
matplotlibin (http://www.scipy.org/). \\

Testiversion voi käynnistää hakemistosta 'TiraLabra/Tiralabra/test'
komennolla '../src/vertaile\_nopeuksia.py'. Toteutusdokumentissä
esitetyt tulokset on saatu ajamalla tuo ohjelma.  \\

Ohjelma lukee syötteet tiedostoista 'alkuarvot.txt\_5x5x0',
'alkuarvot.txt\_10x10x0' ja 'alkuarvot.txt\_16x16x0'.
Esimerkiksi 'alkuarvot.txt\_10x10x0' näyttää seuraavalta:\\
\begin{small}
\begin{lstlisting}[frame=single]  % Start your code-block

1e4     #mylambda
1e6     #SD or MC max iter
1e-2    #tolerance for energy convergence
100.0   #temperature [K] for MC
1e-4    #density chance for functional gradient (SD only)
1e-1    #density chance for MC
0.01    #initial step for line minimization in SD
10      #number of grid points in x
10      #number of grid points in y
0       #number of grid points in z (0==2d case, no this dimension)
4.0     #total electron charge (always positive)
3       #number of nuclei
2 2 0 1.0 #grid positin of a nucleus and its charge (always positive)
2 4 0 1.0 #grid positin of a nucleus and its charge (always positive)
8 7 0 2.0 #grid positin of a nucleus and its charge (always positive)
\end{lstlisting}
\end{small}
Ohjelma tuottaa matplotlib kuvan konvergoituneesta elektronitiheydestä. 

%----------------------------------------------------------------------------------------

\section{Javadocia vastaava dokumentaatio}
Javadocia vastaava dokumentaatio on toteutettu Sphinx:illä.  Se
käynnistyy selaimella esim. komennolla 'firefox
TiraLabra/Docs/html/index.html'
\end{document}
