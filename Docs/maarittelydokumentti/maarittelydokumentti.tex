\documentclass[a4paper,12pt, titlepage]{article}
\usepackage{amssymb,amsthm,amsmath} %ams
\usepackage[finnish]{babel} %suomenkielinen tavutus
\usepackage[T1]{fontenc} %skanditavutus
\usepackage[utf8]{inputenc} % skandit utf-8 koodauksella

\usepackage{graphicx} %dokumentti sisältää eps-muotoisia kuvia

\usepackage{changepage}
\usepackage{array}
\usepackage{tabularx}

\usepackage{hyperref}
\hypersetup{
    colorlinks=false,
    pdfborder={0 0 0},
}

\usepackage{float}
\restylefloat{table}


\linespread{1.24} %riviväli 1.5
\sloppy % Vähentää tavutuksen tarvetta, "leventämällä" rivin keskellä olevia välilyöntejä.
\begin{document}

\begin{titlepage}
    \begin{center}
        \vspace*{1cm}
        
        \LARGE
        \textbf{Määrittelydokumentti}
        
        \vspace{0.5cm}
        \Large
        Aineopintojen harjoitustyö: Tietorakenteet ja algoritmit (alkukesä)
        
        \vspace{1.5cm}
        
        \large
        \textbf{Sami Korhonen} \\
        \text{014021868} \\
        \text{sami.korhonen@helsinki.fi}
        
		\vfill        
        \normalsize
        Tietojenkäsittelytieteen laitos\\
        Helsingin yliopisto\\
		\large        
        \today
        
    \end{center}
\end{titlepage}


%\tableofcontents
%\newpage

\section*{Työn aihe}
\subsection*{Ongelma ja tavoite}
Rahtifirma NopsaToimitus haluaa optimoida konttikuljetuksissa käytettävän tilan. Oletettavasti kontin tila on käytetty mahdollisimman tehokkaasti kun \newline
a) kontin lattian pinta-alaa on käytetty mahdollisimman vähän \newline
b) kontin, tai täytetyn lattia-alueen tilavuudessa on mahdollisimman vähän tyhjää\newline

\noindent
Tavoitteena on kehittää mahdollisimman nopea ja pätevä algoritmi, joka selvittää sopivan tavan pakata annetut laatikot annettuun konttiin.

\subsection*{Määritelmiä}
\subsubsection*{Yksiköt ja koordinaatisto}
Työssä käytetään yksikköinä kokonaisluvuiksi pyöristettyjä senttimetrejä, sillä todellisuudessa laatikoiden mitat eivät ole esimerkiksi puristuvuuden vuoksi tämän tarkempia. \newline

\noindent
Koska kontti ja laatikot ovat suorakulmaisia särmiöitä, voidaan käyttä kolmiulotteista karteesista koordinaatistoa. Tässä työssä koordinaatiston akseleita merkitään kirjaimin x, y ja z.

\subsubsection*{Laatikot}
Laatikoiden asettamiseen liittyy muutamia sääntöjä:
\begin{enumerate}
	\item Laatikon tulee olla kokonaan kontin sisäpuolella
	\item Laatikot eivät saa olla limittäin toistensa kanssa
	\item Laatikon tulee olla kontin seinien suuntaisesti
	\item Laatikko tulee olla tuettu koko pohjaltaan
\end{enumerate}


\subsubsection*{Pakkaaminen}
Toistaiseksi työssä ei kiinnitetä huomiota seuraaviin seikkoihin:

\begin{enumerate}
	\item Samaa tavaraa sisältävät laatikot tulisi asettaa vierekkäin
	\item Rahdin tulisi olla mahdollisimman helppo pakata ja purkaa
\end{enumerate}

\section*{Ratkaisu}
\subsection*{Algoritmit ja tietorakenteet}

\section*{Syötteet ja käyttäminen}
\subsection*{Syötteet}
Ohjelmalle syötetään kontin mitat alustavasti komentorivipohjaisella käyttöliittymällä, mutta ohjelmaan voidaan toteuttaa myöhemmin graafinen käyttöliittymä. Laatikoiden koot ja lukumäärät voidaan syöttää käyttöliittymän kautta tai lukemalla tiedot tiedostosta.

\subsection*{Käyttäminen}
Ohjelman käyttö tapahtuu yksinkertaisen komentorivipohjaisen tai graafisen käyttöliittymän avulla. Lisäksi ohjelma esittää mahdollisesti tuotetun lastausjärjestelmän kuvina ja/tai kenties 3D-graafisesti.

\section*{Aika- ja tilavaativuudet}

\section*{Lähteet}






Määrittele mikä on tavote
- käyttää mahdollisimman vähän lavametrejä ? 
- mahollisimman vähän hukkakuutioita ?


\end{document}